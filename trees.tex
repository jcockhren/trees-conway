\documentclass{beamer}
\usepackage{tikz}
\usetikzlibrary{arrows}
\title{Tree Data Structures}
\tikzset{
    nb/.style = {treenode, circle, black, draw=black},
    invisible/.style={opacity=0},
    visible on/.style={alt=#1{}{invisible}},
    alt/.code args={<#1>#2#3}{%
        \alt<#1>{\pgfkeysalso{#2}}{\pgfkeysalso{#3}}
    }
}
\begin{document}
    \frame{\titlepage}
    \frame{
        \frametitle{Tree Data Structures}
        \framesubtitle{Properties of Trees}
        \begin{enumerate}
            \item<1-> One node is distinguished as the root.
            \item<2-> Every node $c$, except root, is connected by an edge from exactly one other node $p$. Node $p$ is $c$'s $parent$, and $c$ is one of $p$'s $children$. 
            \item<3-> The path from the root to any node is unique. The number of edges traverse is called the $path$ $length$.
        \end{enumerate}
    }
    \frame{
        \frametitle{Tree Data Structures}
        \framesubtitle{Yeah And...}
        \begin{itemize}
            \item In a $Binary$ $Tree$ no node can have more than two children.
            \item There are a special class of trees called $Binary$ $Search$ $Trees$.
        \end{itemize}
    }
    \frame{
        \frametitle{Tree Data Structures}
        \framesubtitle{But Why Trees?}
        Because finding things in LinkedLists is slooooow when the lists are large.
        \newline
        \newline
        \small{$Hint$: Remember the main difference between arrays and linked lists.}
    }
    \frame{
        \frametitle{Tree Data Structures}
        \framesubtitle{Examples of Binary Trees}
        \begin{itemize}
            \item Filesystems
            \item Expression Trees (Wut?)
            \item Indexing/Finding stuff
        \end{itemize}
    }
    \frame{
        \frametitle{Tree Data Structures}
        \framesubtitle{Binary Tree Insertion}
        \subsection*{Binary Search Tree order Property}
        For every node $X$, all nodes on the left subtree have smaller values than $X$, and all nodes in the right subtree have values larger than $X$.
    }
    \frame{
        \frametitle{Tree Data Structures}
        \framesubtitle{Binary Tree Traversal}
        \begin{itemize}
            \item Breadth First 
            \item Depth First
            \item Binary Search
        \end{itemize}
    }
    \frame{
        \frametitle{Tree Data Structures}
        \framesubtitle{Binary Tree Traversal}
        Find 38, using the three tree traversal methods.
        \begin{center}
        \begin{tikzpicture}[->,>=stealth,level/.style={sibling distance = 4cm/#1,level distance = 1.5cm}]
            \node[nb]{33}
                child {node[nb]{15}
                    child {node[nb]{10}
                    }
                    child {node[nb]{20}
                    }
                }
                child {node[nb]{47}
                    child {node[nb]{38}
                        child {node[nb]{36}
                        }
                        child[invisible] {node[nb]{39}
                        }
                    }
                    child {node[nb]{51}
                    }
                }
        \end{tikzpicture}
        \end{center}
    }
    \frame{
        \frametitle{Tree Data Structures}
        \framesubtitle{Binary Tree Insertion}
        Insert 39
        \begin{center}
        \begin{tikzpicture}[->,>=stealth,level/.style={sibling distance = 4cm/#1,
          level distance = 1.5cm}]
            \node[nb]{33}
                child {node[nb]{15}
                    child {node[nb]{10}
                    }
                    child {node[nb]{20}
                    }
                }
                child {node[nb]{47}
                    child {node[nb]{38}
                        child {node[nb]{36}
                        }
                        child[visible on=<2->] {node[nb]{39}
                        }
                    }
                    child {node[nb]{51}
                    }
                }
        \end{tikzpicture}
        \end{center}
    }
    \frame{
        \frametitle{Tree Data Structures}
        \framesubtitle{Recursion}
        \begin{itemize}
            \item<1-> What is recursion?
            \item<2-> What are some examples of recursion in nature?
            \item<3-> How do you caclulate binary tree height recursively?  
        \end{itemize}
    }
    \frame{}
    \frame{
        \begin{center}
        \huge{Conway's Game of Life}
        \newline
        \newline
        \large{A Dive into Test Driven Development}
        \end{center}
    }
    \frame{
        \frametitle{Conway's Game of Life}
        \framesubtitle{Our Goal}
        \begin{itemize}
            \item Learn how to work in a small team.
            \item Learn how to digest a problem with little guidance.
            \item Learn how to use testing to solve a problem.
        \end{itemize}
    }
    \frame{
        \frametitle{Conway's Game of Life}
        \framesubtitle{Get the Files}
        https://github.com/jcockhren/gameoflife
    }
    \frame{
        \frametitle{Conway's Game of Life}
        \framesubtitle{The $How$ and $What$ of Testing}
        \begin{itemize}
            \item What does a testing suite look like?
            \item What should you test?
            \item What things should one consider before writing your first test?
            \item How many asserts should you have per test? 
        \end{itemize}
    }
    \frame{
        \frametitle{Conway's Game of Life}
        \framesubtitle{The Rules}
        \begin{enumerate}
            \item Any live cell with fewer than two live neighbours dies, as if caused by under-population.
            \item Any live cell with two or three live neighbours lives on to the next generation.
            \item Any live cell with more than three live neighbours dies, as if by overcrowding.
            \item Any dead cell with exactly three live neighbours becomes a live cell, as if by reproduction.
        \end{enumerate}
    }
    \frame{
        \frametitle{Conway's Game of Life}
        \framesubtitle{Rules of Engagement}
        \begin{itemize}
            \item Work in pairs
            \item You have 45 minutes per session
            \item Erase all your code at the end of each session and switch partners.
            \item no gems
            \item Internet solutions are a no-no
        \end{itemize}
    }
    \frame{
        \frametitle{Conway's Game of Life}
        \framesubtitle{How to Win}
        \begin{itemize}
            \item Implement Conway's rules
            \item Verify correctness using 2 Still Lifes and 2 Oscillators (i.e. passing tests). 
        \end{itemize}
    }
    \frame{
        \frametitle{Conway's Game of Life}
        \framesubtitle{Bootstrap}
        \begin{itemize}
            \item Focus on how you want to store the data.
            \item How will you define a live cell?
            \item How will you define a 'world'? 
        \end{itemize}
    }
    \frame{
        \frametitle{Slides Available Below}
        https://github.com/jcockhren/trees-conway
    }
\end{document}
